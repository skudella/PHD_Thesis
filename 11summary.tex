\acresetall
\chapter{Summary and conclusion}\label{cha:summary}
Performance upgrades of colliders always require the development of new particle detectors, to deal with the new challenges coming with such an upgrade. For the \ac{CMS} experiment, the upcoming upgrades of the \ac{LHC} to collision energies of $14\,\rm{TeV}$ and instantaneous luminosities of $\mathcal{L}=2\cdot10^{34}\,\rm{cm^{-2}s^{-1}}$ bring several challenges concerning the particle flux and radiation damage in the detector. To deal with them, the pixel detector of the \ac{CMS} experiment needs to be exchanged with an upgraded version (\ac{CMS} Phase I Upgrade). The fourth layer of the new pixel detector will be produced by a German consortium including \ac{KIT}.\\
\\
Pixel detectors provide high granularity and good spatial resolution of particle positions, but hybrid sensor-readout solutions require a high density interconnection technology. The bump bonding interconnection technology is able to provide such a density as well as a good mechanical connection between \ac{ROC} and sensor.

Typical bump bonding technologies are based on lithographic bump deposition processes of for example indium or tin-lead solder bumps. Such bump bonding processes are used for the production of the new \ac{CMS} pixel detector for the Phase I Upgrade. Half of the new fourth pixel layer will be produced at \ac{KIT}, which has decided to use RTI International as the external vendor for the bumping (see sec.~\ref{sec:RTI_bumping}). The bonding will be performed in-house using a Finetech flip-chip bonder that performs both chip placement and reflow process (see sec.~\ref{sec:SnPb_flip-chip}).

For the production of the pixel detector modules for the \ac{CMS} Phase I Upgrade, a cleaning procedure was developed to remove the thick photo-resist protection layer from RTI on top of the bumps (see ch.~\ref{cha:cleaning}) placed there to protect the bumps during transport. The cleaning procedure is based on chemical baths of acetone, isopropyl alcohol and water. To handle the bumped \ac{ROC}s during the cleaning process, special mechanics were designed, including a cleaning tray, a vacuum system, and a grid which keeps the chips from floating. The cleaning process was tuned and the inspections by scanning electron microscope and energy dispersive X-ray spectroscopy showed that there are no photo-resist residuals larger than $1\,\si{\micro \meter}$ left on the bumps or the chip surface after cleaning.
\\
\\
The standard lithographic bump deposition processes have the disadvantage that they can only be performed on wafer level and are therefore cost-intensive and lack flexibility, especially in the R$\&$D phase of new particle detectors. For this reason, new bump bonding technologies that allow cheaper and more flexible development of pixel detectors are being investigated. 

\ac{KIT} has decided to develop a gold-stud bump bonding process parallel to the production of the new pixel detector. The gold-stud bump bonding technology is a possible candidate for the development of new pixel detectors. Here, a ball wire bonder places gold-stud bumps onto metal pads by thermo-ultrasonic bonding. The flip-chip bonding process is then performed by a flip-chip bonder using thermocompression. Therefore, the gold-stud bump bonding process does not require any lithography at all.\\
\\
The gold-stud bumping process at \ac{KIT} is performed using a Kulicke $\&$ Soffa IConn ball wire bonder. After a detailed investigation of the process sequence, the bonding parameters and their influence on the mechanical strength and the shape of the gold-stud bumps were studied (see sec.~\ref{sec:IConn_working principle}).

To optimize the bump connection to the aluminum surface of the metal pads, a region of high mechanical strength and stability was determined by a systematic variation of the bonding parameters. This study lead to a stable region around a bonding force of $9\,\rm{g}$, an \acs{USG}\footnote{\acf{USG}} current of $30\,\rm{mA}$ and a bonding time of $8\,\rm{ms}$, allowing the placement of $30\,\si{\micro \meter}$ bumps with a mechanical strength of $8.7\,\rm{g}$ (see sec.~\ref{sec:mechanical_strength_optimization}). To place bumps on small passivation openings of $15\,\si{\micro \meter}$, the parameters needed to be adjusted towards larger \acs{USG} power, which lead to bump diameters of $35\,\si{\micro \meter}$ on passivation openings of $15\,\si{\micro \meter}$ or $18\,\si{\micro \meter}$.

In a second optimization step, the long-term stability of the bumping process was iteratively tuned. Therefore, the parameters defining the wire shear process were varied to provide a correct wire shear and wire re-feeding process. This optimization allowed placing more than $4000$ gold-stud bumps without any interruption of the process (see sec.~\ref{sec:long-term_stability}) which corresponds to the bumping of a full \ac{CMS} pixel single sensor or \ac{CMS} pixel \ac{ROC} within five minutes.

During the process optimization an unknown error occurred. The so-called ``wire-ripping and sparking error'' (see sec.~\ref{sec:wrse}) caused the ball wire bonder to apply a voltage of $5000\,\rm{V}$ to a wire connected to a gold-stud bump. This destroyed almost all the electric circuits on the chip. The error was solved by using a newer and harder wire. To protect the material from the EFO voltage in the future, an insulation layer made of ceramics was designed. It is placed beneath the chip to insulate it from the ground potential of the bonding table. Although the insulation layer cannot protect the \acl{PUC} hit by the high voltage pulse, it does protect the rest of the chip.
\\
\\As another application for gold-stud bumps, a gold-stud bump UBM layer was designed. This layer can be used to create a solder wettable gold surface for power electronics. The bonding parameters were tuned to create very flat bumps with a thickness of only $6\,\si{\micro \meter}$. In the context of this application, the vacuum jig of the bonder has been improved by designing an adapter plate that allows fixing smaller dies (see sec.~\ref{sec:UBM_layer}). 
\\
\\
Further-more, the flip-chip bonding of gold-stud bumped chips was investigated. The flip-chip bonding at \ac{KIT} is performed with a Finetech FINEPLACER$\rm{\symbTM}$ femto. The goal of the first bonding tests was to reduce the bonding temperature. A reduction of the bonding temperature to $250\,\si{\degreeCelsius}$ for $60\,\rm{s}$ of bonding time was achieved, avoiding any electromigration inside the sensor or chip. Cross-sections through a flip-chip bonded assembly showed a good metallic connection between the gold-stud bumps (see sec.~\ref{sec:flip-chip_bonding_temperature}) but they also showed a systematic misalignment of the flip-chip bonded chips. This is due to the asymmetric and not completely flat top of the gold-stud bumps, which causes horizontal forces during the bonding and a shift of the chips. In a second step, the bonding force was decreased to reduce these horizontal forces during bonding. To provide good connections of all bumps, the bonding force was set to a value of $250\,\rm{N}$ (see sec.~\ref{sec:flip-chip_bonding_force}).

Since the reduction of the bonding force was not sufficient to remove the systematic misalignment, an additional step was introduced into the bump bonding process. In this step, the top of the gold-stud bumps is flattened by pressing the bumped side of the chip onto the bonding table of the femto. The parameters of this procedure were tuned to a bonding force of $400\,\rm{N}$ and a bonding time of $50\,\rm{s}$. The bonding temperature was set to $40\,\si{\degreeCelsius}$ to avoid any connection between bumps and bonding table. The flattening allows creating a smooth and planar surface of gold-stud bumps, drastically decreasing the systematic misalignment of the chips (see sec.~\ref{sec:flattening}).

The assemblies show typical pull forces of approximately $9-10\,\rm{kg}$ ($2.2-2.4\,\rm{g}$ per bump). Electrical tests of the connections were only possible with a radioactive source inducing charge into the sensor bulk. They showed a good electrical connection for most of the bumps. Unfortunately, these tests also showed a rotation of the chips during bonding. To solves this problem, it is necessary to improve the vacuum system of the femto. Since the femto is needed for the production of the pixel detector modules for the CMS Phase I Upgrade, this improvement could not be implemented during this thesis.
\\
\\
In addition to the gold-to-gold process, alternative bump bonding technologies like SnPb solder to gold-stud bumps, \ac{PPS} solder bumps on gold-stud bumps or anisotropic conductive gluing foils were described and partially tested. These techniques also show great potential for the R$\&$D of new pixel detectors (see sec.~\ref{sec:gold-stud_outlook}).
\\
\\Although the investigations on gold-stud bump bonding are very promising, future investigations could even improve the process. On the bumping side of the process, only minor parts are not optimized yet. With the change to a harder wire, a second systematic process optimization of the mechanical strength is needed to find the optimum parameters. Further more, different wire types can be tested to find the ideal bonding wire.

On the flip-chip bonding side, an additional reduction of the bonding temperature could allow the usage of gold-stud bump bonding for irradiation studies. The flattening process can be further improved by pressing the bumps onto a polished glass plate instead of the bonding table, avoiding any asperities in the flattening process. To avoid any systematic misalignment and rotational movement the vacuum jig and the pick-up tool of the femto need an upgrade. With these improvements and upgrades, a fully working gold-stud bump bonding process for \ac{HEP} particle detectors should be easily possible.